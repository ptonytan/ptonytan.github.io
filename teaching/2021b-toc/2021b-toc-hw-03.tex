\documentclass[11pt, a4paper]{article}


\usepackage{../mysty}

\renewcommand{\lesson}{0}
\renewcommand{\lessontitle}{Homework 3}
\renewcommand{\fulltitle}{\lessontitle}

\renewcommand{\themycount}{1.\arabic{mycount}}

\usepackage{xr}
\externaldocument{../lesson-08/2021b-toc-note-08}
%\usetikzlibrary{automata,positioning}
%\tikzset{initial text={}}



%%%%%%%%%%%%%%%%%%%%%%%%%%%%%%%%%%%%%%%%%%%%%%%%%%%%%%%%%%%%%%%%%%%%%%%%%%%%%%%%%%%%%%
%%%%%%% START DOCUMENT %%%%%%%%%%%%%%%%%%%%%%%%%%%%%%%%%%%%%%%%%%%%%%%%%%%%%%%%%%%%%%%
%%%%%%%%%%%%%%%%%%%%%%%%%%%%%%%%%%%%%%%%%%%%%%%%%%%%%%%%%%%%%%%%%%%%%%%%%%%%%%%%%%%%%%

\begin{document}
\date{}



\begin{center}
{\Large {\bf \fulltitle}}
\end{center}
\begin{center}
{\Large {\bf Due on Tuesday, 11:59 am, 30 May 2022 (111/05/30)}}
\end{center}


\paragraph*{Note:} There are 8 questions altogether.
For Questions 1--3, we will use the following notation.
For a function $g:\bbN\to\bbN$, let $\size(g)$ denote
the class of languages such that $L \in \size(g)$ if and only if 
$L$ is decided by a circuit family $\{C_n\}$ such that for sufficiently large $n$:
\begin{eqnarray*}
|C_n| & \leq & g(n)
\end{eqnarray*}
That is, there is $n'$ such that for every $n\geq n'$, $|C_n|\leq g(n)$.


\paragraph*{Question 1.}
\begin{enumerate}[(a)]
\item 
Show that every function $f:\{0,1\}^t\to\{0,1\}$
can be computed by a circuit of size $\leq 3t2^t$.
\item
Show that for every $k\geq 1$, there is a language $L$ such that the following holds.
\begin{enumerate}[(P1)]
\item 
$L \in \size(n^{k+1})$.
\item
For sufficiently large $n$, there is no circuit of size $\leq n^k$ that computes $L\cap\{0,1\}^n$.
\end{enumerate}
Conclude that for every $k\geq 1$, $\size(n^k)\subsetneq \size(n^{k+1})$.
\end{enumerate}
Hint for (b): We know that for every $t$, there is a function $f:\{0,1\}^t\to\{0,1\}$
such that $f$ is not computable by circuit of size $2^t/(10t)$.
Combine this with (a) for some appropriate value~$t$.

\paragraph*{Question 2.}
Prove that for every $k\geq 1$,
there is a language $L \in \sigmap {4}$ that has properties (P1) and (P2) above.
Then, conclude that for every $k\geq 1$, $\sigmap {4}\setminus \size(n^k) \neq \emptyset$.

\vspace{0.2cm}
\noindent
Hint: Consider the language $L$ in Question 1. Then, for every $n$,
consider the ``lexicographically first'' circuit $C_n$ of size $\leq n^{k+1}$
that is not equivalent to any of the circuit of size $\leq n^k$.

\paragraph*{Question 3.}
Prove that for every $k\geq 1$,
there is a language $L \in \sigmap {2} \setminus \size(n^k)$.



\paragraph*{Question 4.}
\begin{itemize}
\item
Let $\cH_{n,k}$ be a pairwise independent collection of hash functions $h:\{0,1\}^\to\{0,1\}^k$.
Prove that for every $x\in \{0,1\}^n$, for every $y \in \{0,1\}^k$,
$\prdist {h\in \cH_{n,k}} { h(x)=y }  =  2^{-k}$.
\item
Prove Theorem~\ref{theo:hash-exists-b}, i.e.,
the collection $\cH_{n,n} \defeq \{h_{A,b} : A \in \{0,1\}^{n\times n}\ \text{and}\ b \in \{0,1\}^{n\times 1}\}$ 
is pair-wise independent.
\end{itemize}


\paragraph*{Question 5.}
Prove that $\class{MA}\subseteq \sigmap{2}$.


\end{document}
%%%%%%%%%%%%%%%%%%%%%%%%%%%%%%%%%%%%%%%%%%%%%%%%%%%%%%%%%%%%%%%%%%%%%%%%%%%%%%%%%%%%%%
%%%%%%% END OF DOCUMENT %%%%%%%%%%%%%%%%%%%%%%%%%%%%%%%%%%%%%%%%%%%%%%%%%%%%%%%%%%%%%%
%%%%%%%%%%%%%%%%%%%%%%%%%%%%%%%%%%%%%%%%%%%%%%%%%%%%%%%%%%%%%%%%%%%%%%%%%%%%%%%%%%%%%%





