\documentclass[11pt, a4paper]{article}

\usepackage{../mysty}

\renewcommand{\lesson}{4}
\renewcommand{\lessontitle}{The polynomial hierarchy}
\renewcommand{\fulltitle}{Lesson \lesson: \lessontitle}

\usepackage{xr}
\externaldocument{../lesson-02/2021b-toc-note-02}

%%%%%%%%%%%%%%%%%%%%%%%%%%%%%%%%%%%%%%%%%%%%%%%%%%%%%%%%%%%%%%%%%%%%%%%%%%%%%%%%%%%%%%
%%%%%%% START DOCUMENT %%%%%%%%%%%%%%%%%%%%%%%%%%%%%%%%%%%%%%%%%%%%%%%%%%%%%%%%%%%%%%%
%%%%%%%%%%%%%%%%%%%%%%%%%%%%%%%%%%%%%%%%%%%%%%%%%%%%%%%%%%%%%%%%%%%%%%%%%%%%%%%%%%%%%%

\begin{document}
\date{}


%\thispagestyle{empty}

\begin{center}
{\Large {\bf \fulltitle}}
\end{center}
\vspace{0.5cm}

\noindent
{\bf Theme:} The polynomial hierarchy.

\vspace{0.5cm}

For every integer $i\geq 1$, the class $\sigmap {i}$ is defined as follows.
A language $L \subseteq \{0,1\}^*$ is in $\sigmap {i}$,
if there is a polynomial $q(n)$ and a polynomial time DTM $\cM$
such that for every $w\in \{0,1\}^*$,
$w\in L$ if and only if the following holds.
\begin{eqnarray}
\label{eq:sigmap}
& 
\exists y_1 \in \{0,1\}^{q(|w|)}\ 
\forall y_2 \in \{0,1\}^{q(|w|)}\
\cdots \
Q y_i \in \{0,1\}^{q(|w|)}\
\cM \ \text{accepts}\ (w,y_1,\ldots,y_i) 
&
\end{eqnarray}
where $Q = \exists$, if $i$ is odd and $Q=\forall$, if $i$ is even.


The class $\pip {i}$ is defined as above, but the sequence of quantifiers in~(\ref{eq:sigmap}) starts with $\forall$.
Alternatively, it can also be defined as
$\pip {i} \defeq \{\overline{L} : L \in \sigmap {i}\}$.
Note that $\npt = \sigmap {1}$ and $\conpt = \pip {1}$.

\begin{remark}
The class $\sigmap {i}$ can also be defined as follows.
A language $L$ is in $\sigmap {i}$, if there is a polynomial time ATM $\cM$ that decides $L$ such that
for every input word $w\in \{0,1\}^*$,
the run of $\cM$ on $w$ can be divided into $i$ layers.
Each layer consists of nodes of the same depth in the run.
(Recall that the run of an ATM is a tree.)
In the first layer all nodes are labeled with existential configurations,
in the second layer with universal configurations, and so on.
It is not difficult to show that this definition is equivalent to the one above.
\end{remark}

The {\em polynomial time hierarchy} (or, in short, {\em polynomial hierarchy}) is defined as the following class.
\begin{eqnarray*}
\pht & \defeq & \bigcup_{i=1}^{\infty} \ \sigmap {i}
\end{eqnarray*}
Note that $\pht \subseteq \ps$.

It is conjectured that 
$\sigmap {1}\ \subsetneq\ \sigmap {2}\ \subsetneq\ \sigmap {3} \ \subsetneq \ \cdots$.
In this case, we say that {\em the polynomial hierarchy does not collapse}.
We say that {\em the polynomial hierarchy collapses},
if there is $i$ such that $\pht = \sigmap {i}$,
in which case we also say that {\em the polynomial hierarchy collapses to level $i$}.


We define the notion of hardness and completeness for each $\sigmap {i}$ as follows. 
For $i\geq 1$, a language $K$ is {\em $\sigmap {i}$-hard}, if for every $L \in \sigmap {i}$, $L \leq_p K$. 
It is {\em $\sigmap {i}$-complete}, if it is in $\sigmap {i}$ and it is $\sigmap {i}$-hard. 
The same notion can be defined analoguously for $\pht$ and each $\pip {i}$.


Define the language $\sigmasat {i}$ as consisting of true QBF of the form:
\begin{eqnarray*}
& & \exists \vx_1 \ \forall \vx_2 \ \cdots \ Q \vx_i\ \varphi(\vx_1,\ldots,\vx_i)
\end{eqnarray*}
where $\varphi(\vx_1,\ldots,\vx_i)$ is quantifier-free Boolean formula and $Q=\exists$, if $i$ is odd, and $Q=\forall$, if $i$ is even.
Here $\vx_1,\ldots,\vx_i$ are all vectors of boolean variables.
In other words, $\sigmasat {i}$ is a subset of $\tqbf$ where the number of quantifier alternation is limited to $(i-1)$.
The language $\pisat {i}$ is defined analogously with the starting quantifiers being $\forall$.

\begin{theorem}
\label{theo:ph}~
\begin{itemize}
\item 
For every $i\geq 1$,
$\sigmasat {i}$ is $\sigmap {i}$-complete and $\pisat {i}$ is $\pip {i}$-complete.
\item
If $\sigmap {i} = \pip {i}$ for some $i\geq 1$, then the polynomial hierarchy collapses.
\item
If there is language that is $\pht$-complete, then the polynomial hierarchy collapses.
\end{itemize}

\end{theorem}





\end{document}

%%%%%%%%%%%%%%%%%%%%%%%%%%%%%%%%%%%%%%%%%%%%%%%%%%%%%%%%%%%%%%%%%%%%%%%%%%%%%%%%%%%%%%
%%%%%%% END OF DOCUMENT %%%%%%%%%%%%%%%%%%%%%%%%%%%%%%%%%%%%%%%%%%%%%%%%%%%%%%%%%%%%%%
%%%%%%%%%%%%%%%%%%%%%%%%%%%%%%%%%%%%%%%%%%%%%%%%%%%%%%%%%%%%%%%%%%%%%%%%%%%%%%%%%%%%%%





