\documentclass[11pt, a4paper]{article}

\usepackage{../mysty}

\renewcommand{\lesson}{11}
\renewcommand{\lessontitle}{Interactive proofs}
\renewcommand{\fulltitle}{Lesson \lesson: \lessontitle}

\usepackage{xr}
\externaldocument{../lesson-08/2022b-cot-note-08}



%%%%%%%%%%%%%%%%%%%%%%%%%%%%%%%%%%%%%%%%%%%%%%%%%%%%%%%%%%%%%%%%%%%%%%%%%%%%%%%%%%%%%%
%%%%%%% START DOCUMENT %%%%%%%%%%%%%%%%%%%%%%%%%%%%%%%%%%%%%%%%%%%%%%%%%%%%%%%%%%%%%%%
%%%%%%%%%%%%%%%%%%%%%%%%%%%%%%%%%%%%%%%%%%%%%%%%%%%%%%%%%%%%%%%%%%%%%%%%%%%%%%%%%%%%%%

\begin{document}
\date{}


%\thispagestyle{empty}

\begin{center}
{\Large {\bf \fulltitle}}
\end{center}
\vspace{0.5cm}

\noindent
{\bf Theme:} The class $\ip$, $\ma$ and $\am$.

\section{The class $\ip$}

Let $\Sigma=\{0,1\}$ and $\#$ be a symbol.
Let $P:(\Sigma\cup\{\#\})\to\Sigma^*$ be an {\em arbitrary} function.
Let $V$ be a probabilistic TM whose inputs are of the form:
$$
w\ \#\ u_1\ \#\ v_1\ \#\ u_2\ \# \ v_2\ \#\ \cdots \ \# \ u_{m}\ \# \ v_{m} 
$$
where $w,u_1,v_1,\ldots,u_m,v_m$ are all strings from $\Sigma^*$.
The outcome of $V$ can be {\em accept}, {\em reject} or ``send a string $u$ to $P$.''

The function $P$ is usually called the {\em prover} and $V$ the {\em verifier}.
The interaction between $P$ and $V$, denoted by $(P,V)$, on input word $w\in \Sigma^*$ consists of rounds defined as follows.

\vspace{0.3cm}
\noindent
(Round~1:)
\begin{itemize}
\item
Run $V$ on $w$.

If it accepts/rejects, then the interaction stops.

If the outcome is ``sends a string $u_1$ to $P$,''
then $V$ sends $u_1$ to $P$.

\item 
Let $P(w\#u_1)=v_1$.

Then, $P$ sends $v_1$ to $V$, and the interaction continues to round~2.
\end{itemize}
(Round~2:)
\begin{itemize}
\item
Run $V$ on $w\#u_1\#v_1$.

If it accepts/rejects, then the interaction stops.

If the outcome is ``sends a string $u_2$ to $P$,''
then it sends $u_2$ to $P$.

\item 
Let $P(w\#u_1\#v_1\#u_2)=v_2$.

Then, $P$ sends $v_2$ to $V$, and the interaction continues to round~3.
\end{itemize}
and so on. The interaction continues until $V$ accepts/rejects, in which case
we say that the interaction $(P,V)$ accepts/rejects $w$.

On each round $i$, the verifier $V$ starts with its initial state
and the position of its head is on the first position of $w\# u_1 \# v_1 \#\cdots  \#  u_{i-1}\# v_{i-1}$.
On each round $i$, we call the string $u_i$ {\em the verifier's query} 
and $v_i$ {\em the prover's reply}.

\begin{remark}
We usually assume that $V$ runs in polynomial time in {\em the length of the input word $w$}.
That is, there is a polynomial $p(n)$ such that on each round $i$
the run time of $V$ on $w\# u_1 \# v_1 \#\cdots  \#  u_{i-1}\# v_{i-1}$
is bounded by $p(|w|)$.

In this case we may assume that $V$ always tosses the random string $r$ before round~1 starts
and in each round $i$, the verifier $V$ is a deterministic TM with input 
$(w,r)\# u_1 \# v_1 \#\cdots  \#  u_{i-1}\# v_{i-1}$.
Moreover, the length of each reply $v_i$ is also bounded by the $p(|w|)$
and so is the number of rounds in the interaction.  

Note also that the prover $P$ does not know the random string $r$.
He only knows the input word and the queries sent by the verifier. 
\end{remark}

\begin{definition}
\label{def:verifier-decide}
A (polynomial time) verifier $V$ decides a language $L$,
if for every word $w\in \Sigma^*$, the following holds.
\begin{itemize}
\item
If $w\in L$, then there is a prover $P$ such that $\prarg {(P,V)\ \text{accepts}\ w} \geq 2/3$.
\item
If $w\notin L$, then for every prover $P$, $\prarg {(P,V)\ \text{accepts}\ w} \leq 1/3$.
\end{itemize}
Here it is useful to recall that $V$ is a probabilistic polynomial time TM.
\end{definition}

The class $\ip$ is defined as:
$$
\ip \ \defeq \ \{L \mid \text{there is a polynomial time verifier}\ V \ \text{that decides}\ L\}
$$




\begin{example}
We will consider the interactive proofs for following two languages.
\begin{itemize}
\item
$\noniso \defeq \{ (G_0,G_1) \mid \text{$G_0$ is not isomorphic to $G_1$}\}$.
\item
$\nonsq \defeq \{ (a,n) \mid \text{$a$ and $n$ are integers and $a \not\equiv b^2 \pmod n$, for every integer $b$}\}$.
\end{itemize}
\end{example}


\begin{lemma}
\label{lem:ip-ps}
$\ip\subseteq \ps$.
\end{lemma}


\section{The class $\ma$ and $\am$}

\paragraph*{The class $\am$.}
The {\em Arthur-Merlin} ($\am$) class is defined as the class $\ip$
with additional restrictions.
On input $w$, it does the following.
\begin{itemize}
\item
$V$ generates a random string $r$ and sends it to $P$.
\item 
$P$ replies with a string $p$.
\item 
$V$ runs a deterministic computation on input $w,r,p$.

That is, $V$ is not allowed to use any random string except $r$.
\end{itemize}


\paragraph*{The class $\ma$.}
The {\em Merlin-Arthur} ($\ma$) class is defined as the class $\ip$
with additional restrictions.
On input $w$, it does the following.
\begin{itemize}
\item 
$P$ sends a string $p$ to $V$.
\item 
Run $V$ on input $w,p$, where $V$ is a polynomial time PTM.

Here $V$ is allowed to generate some random string.
\end{itemize}

Note that in the class $\am$ and $\ma$ the interaction consists of only one round.
It can be easily generalized to multiple rounds.



\begin{theorem}
\label{theo:am-ma-ph}~
\begin{itemize}
\item 
$\am\subseteq \sigmap {3}$.
\item
$\ma\subseteq \sigmap {2}$.
\end{itemize}
\end{theorem}

Theorem~\ref{theo:am-ma-ph} can be proved using the same technique as Theorem~\ref{theo:bpp-ph}.


\end{document}

%%%%%%%%%%%%%%%%%%%%%%%%%%%%%%%%%%%%%%%%%%%%%%%%%%%%%%%%%%%%%%%%%%%%%%%%%%%%%%%%%%%%%%
%%%%%%% END OF DOCUMENT %%%%%%%%%%%%%%%%%%%%%%%%%%%%%%%%%%%%%%%%%%%%%%%%%%%%%%%%%%%%%%
%%%%%%%%%%%%%%%%%%%%%%%%%%%%%%%%%%%%%%%%%%%%%%%%%%%%%%%%%%%%%%%%%%%%%%%%%%%%%%%%%%%%%%





